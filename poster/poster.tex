\documentclass[landscape,final,a0paper,fontscale=0.285]{baposter}
%% \documentclass[landscape,final,a0paper,fontscale=0.4]{baposter}
%% \documentclass[landscape,final,a0paper,fontscale=0.35]{baposter}

\usepackage{calc}
\usepackage{graphicx}
\usepackage{amsmath}
\usepackage{amssymb}
\usepackage{relsize}
\usepackage{multirow}
\usepackage{rotating}
\usepackage{bm}
\usepackage{url}

\usepackage{graphicx}
\usepackage{multicol}

%% \usepackage{times}
%% \usepackage{helvet}
%\usepackage{bookman}
\usepackage{palatino}

\usepackage{lipsum}
%% \usepackage{jabbrv}
%% \usepackage{setspace}
\usepackage{enumitem}

\newcommand{\captionfont}{\footnotesize}

\graphicspath{{../paper/figures/}}
%% \graphicspath{{images/}{../images/}}
%% \usetikzlibrary{calc}

%%%%%%%%%%%%%%%%%%%%%%%%%%%%%%%%%%%%%%%%%%%%%%%%%%%%%%%%%%%%%%%%%%%%%%%%%%%%%%%%
%%%% Some math symbols used in the text
%%%%%%%%%%%%%%%%%%%%%%%%%%%%%%%%%%%%%%%%%%%%%%%%%%%%%%%%%%%%%%%%%%%%%%%%%%%%%%%%

%%%%%%%%%%%%%%%%%%%%%%%%%%%%%%%%%%%%%%%%%%%%%%%%%%%%%%%%%%%%%%%%%%%%%%%%%%%%%%%%
% Multicol Settings
%%%%%%%%%%%%%%%%%%%%%%%%%%%%%%%%%%%%%%%%%%%%%%%%%%%%%%%%%%%%%%%%%%%%%%%%%%%%%%%%
\setlength{\columnsep}{1.5em}
\setlength{\columnseprule}{0mm}

%%%%%%%%%%%%%%%%%%%%%%%%%%%%%%%%%%%%%%%%%%%%%%%%%%%%%%%%%%%%%%%%%%%%%%%%%%%%%%%%
% Save space in lists. Use this after the opening of the list
%%%%%%%%%%%%%%%%%%%%%%%%%%%%%%%%%%%%%%%%%%%%%%%%%%%%%%%%%%%%%%%%%%%%%%%%%%%%%%%%
\newcommand{\compresslist}{%
\setlength{\itemsep}{1pt}%
\setlength{\parskip}{0pt}%
\setlength{\parsep}{0pt}%
}

%%%%%%%%%%%%%%%%%%%%%%%%%%%%%%%%%%%%%%%%%%%%%%%%%%%%%%%%%%%%%%%%%%%%%%%%%%%%%%
%%% Begin of Document
%%%%%%%%%%%%%%%%%%%%%%%%%%%%%%%%%%%%%%%%%%%%%%%%%%%%%%%%%%%%%%%%%%%%%%%%%%%%%%

\begin{document}

%%%%%%%%%%%%%%%%%%%%%%%%%%%%%%%%%%%%%%%%%%%%%%%%%%%%%%%%%%%%%%%%%%%%%%%%%%%%%%
%%% Here starts the poster
%%%---------------------------------------------------------------------------
%%% Format it to your taste with the options
%%%%%%%%%%%%%%%%%%%%%%%%%%%%%%%%%%%%%%%%%%%%%%%%%%%%%%%%%%%%%%%%%%%%%%%%%%%%%%
% Define some colors

%\definecolor{lightblue}{cmyk}{0.83,0.24,0,0.12}
%% \definecolor{lightblue}{rgb}{0.145,0.6666,1}

%% \definecolor{headerfade}{rgb}{0.89,0.68,0}
\definecolor{headerfade}{rgb}{0.93, 0.80, 0.38}

\hyphenation{resolution occlusions}
%%
\begin{poster}%
  % Poster Options
  {
  % Show grid to help with alignment
  grid=false,
  columns=4,
  % Column spacing
  colspacing=1em,
  % Color style
  bgColorOne=white,
  bgColorTwo=white,
  borderColor=headerfade,
  headerColorOne=black,
  headerColorTwo=headerfade,
  headerFontColor=white,
  boxColorOne=white,
  boxColorTwo=headerfade,
  % Format of textbox
  textborder=roundedleft,
  % Format of text header
  eyecatcher=true,
  headerborder=closed,
  headerheight=0.17\textheight,
%  textfont=\sc, An example of changing the text font
  headershape=roundedright,
  headershade=shadelr,
  headerfont=\Large\bf\textsc, %Sans Serif
  textfont={\setlength{\parindent}{1.5em}},
  boxshade=plain,
%  background=shade-tb,
  background=plain,
  linewidth=2pt
  }
  % Eye Catcher
  {
    \begin{tabular}{c}
      \relax\\
      \includegraphics[height=13em]{brain.png}
    \end{tabular}
  }
  % Title
  {
    \textbf{\textsc{Selective Processing for\\[0.2em] Real-time Stereo Matching}}\vspace{0.5em}
  }
  % Authors
  {
    \textsc{Eric Hunsberger, Jeff Orchard, Bryan Tripp}\\[0.2em]
    \{ehunsber, jeff.orchard, bptripp\}@uwaterloo.ca\\[0.4em]
    %% Centre for Theoretical Neuroscience, University of Waterloo (\url{http://ctn.uwaterloo.ca})
  }
  % University logo
  {
    %% \includegraphics[height=9.0em]{ctn.png}
    \begin{tabular}{cc}
      \relax\\
      \includegraphics[height=6.0em]{uwaterloo.png} &
      \includegraphics[height=5.0em]{ctn.png}\\
      \multicolumn{2}{c}{
        \includegraphics[height=5.0em]{cnrg.png}}
    \end{tabular}
  }

%%%%%%%%%%%%%%%%%%%%%%%%%%%%%%%%%%%%%%%%%%%%%%%%%%%%%%%%%%%%%%%%%%%%%%%%%%%%%%
%%% Now define the boxes that make up the poster
%%%---------------------------------------------------------------------------
%%% Each box has a name and can be placed absolutely or relatively.
%%% The only inconvenience is that you can only specify a relative position
%%% towards an already declared box. So if you have a box attached to the
%%% bottom, one to the top and a third one which should be in between, you
%%% have to specify the top and bottom boxes before you specify the middle
%%% box.
%%%%%%%%%%%%%%%%%%%%%%%%%%%%%%%%%%%%%%%%%%%%%%%%%%%%%%%%%%%%%%%%%%%%%%%%%%%%%%
    %
    % A coloured circle useful as a bullet with an adjustably strong filling
    \newcommand{\colouredcircle}{%
      \tikz{\useasboundingbox (-0.2em,-0.32em) rectangle(0.2em,0.32em); \draw[draw=black,fill=headerfade,line width=0.03em] (0,0) circle(0.18em);}}

    \newenvironment{blockitemize}
    {%
      \noindent\hspace{-1em}
      \begin{minipage}{\columnwidth + 0em}
      \begin{itemize}
        \setlength{\itemsep}{0.3em}

        \raggedright
    }{
      \end{itemize}
      \end{minipage}
    }

    %%%%%%%%%%%%%%%%%%%%%%%%%%%%%%%%%%%%%%%%%%%%%%%%%%%%%%%%%%%%%%%%%%%%%%%%%%%%%%%%

    \headerbox{Introduction}{name=pane00,column=0,row=0}{
      \begin{blockitemize}
        \item Multiscale Belief Propagation (BP) is an accurate
          stereo-matching algorithm.
        \item Our goal is to reduce computation time while maintaining accuracy
          by focusing computation in important regions of the image.
        \item We also want to incorporate past data into the depth estimate
          for the current frame.
      \end{blockitemize}
    }

    \headerbox{Multiscale BP}{name=pane01,column=0,below=pane00}{
      \noindent
      Belief Propagation (BP) poses the stereo-matching problem in terms of
      the solution to a Markov Random Field. The ``loopy'' BP algorithm then
      uses a message-passing scheme to perform approximate inference on this
      field, resulting in probabilities across all possible disparity matches
      of all the pixels in the scene.

      Multiscale BP reduces the number of iteration \cite{Felzenszwalb2006a}
      TODO
    }

    \headerbox{References}{name=pane02,column=0,below=pane01,above=bottom}{
      \scriptsize
      \renewcommand{\refname}{\vspace{-0.8em}}
      \bibliographystyle{ieeetr}
      %% \bibliographystyle{jabbrv_ieeetr}
      \bibliography{../paper/fast-stereo-eric}
    }

    \headerbox{Fovea Example}{name=pane10,column=1,row=0}{
      \begin{center}
        %% \includegraphics[width=0.75\columnwidth,clip=true,trim=50px 55px 50px 55px]{fovea-examples.png}
        \includegraphics[width=1.0\columnwidth]{fovea-examples-tight.png}
      \end{center}
      %% The fovea can help remove objects
      TODO
    }

    \headerbox{Coarse vs Fine BP}{name=pane11,column=1,row=0,below=pane10}{
      \begin{center}
        \includegraphics[width=1.0\columnwidth]{fovea-rationale.png}
      \end{center}
      Comparing runtimes of BP at coarse (squares) and fine (circles) resolutions
      across different numbers of iterations,
      we see that using more iterations to coarse BP quickly saturates in performance,
      whereas adding iterations to fine BP quickly makes the runtime too long for real time.
      By applying fine BP selectively to important regions,
      we can achieve real-time BP (at 10 Hz) with better performance than coarse BP.
    }

    \headerbox{Importance}{name=pane20,column=2,row=0}{
      \begin{center}
        %% \includegraphics[width=1.0\columnwidth]{importance.png}
        \includegraphics[width=0.6\columnwidth,clip=true,trim=130px 15px 130px 15px]{importance.png}
      \end{center}
      To choose regions of interest for foveation,
      we compare the current coarse depth estimate (middle)
      with the average depth estimate across many frames (top).
      The difference highlights regions that are unexpectedly close (bottom).
    }

    \headerbox{Foveation sequence}{name=pane21,column=2,row=0,below=pane20}{
      TODO
      \begin{center}
        %% \includegraphics[width=1.0\columnwidth,clip=true,trim=220px 100px 200px 120px]{foveation-sequence.png}
        \includegraphics[width=1.0\columnwidth]{foveation-sequence-tight.png}
      \end{center}
    }

    \headerbox{Something}{name=pane30,column=3,row=0}{
    }

    \headerbox{Conclusions}{name=pane31,column=3,row=0,below=pane30}{
      \begin{blockitemize}
        \item TODO
        %% \item Computing the fine data cost and downsampling has
      \end{blockitemize}
    }

    \headerbox{Future Work}{name=pane32,column=3,row=0,below=pane31}{
      \begin{blockitemize}
        \item Use coarser resolution outside fovea to further reduce computation
        \item Use multiple foveas to target multiple regions of interest
      \end{blockitemize}
    }

\end{poster}
\end{document}

%%  LocalWords:  headerfade rgb colspacing bgColorOne bgColorTwo TODO
%%  LocalWords:  borderColor headerColorOne headerColorTwo textborder
%%  LocalWords:  headerFontColor boxColorOne boxColorTwo roundedleft
%%  LocalWords:  eyecatcher headerborder headerheight headershape
%%  LocalWords:  roundedright headershade shadelr headerfont textfont
%%  LocalWords:  boxshade linewidth uwaterloo
%%  LocalWords:  rasters multi Tripp jeff bptripp Multiscale
